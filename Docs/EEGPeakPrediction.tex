\documentclass{article}

\usepackage{amsmath, amsthm, amssymb, amsfonts}
\usepackage{thmtools}
\usepackage{graphicx}
\usepackage{setspace}
\usepackage{geometry}
\usepackage{float}
\usepackage{hyperref}
\usepackage[utf8]{inputenc}
\usepackage[english]{babel}
\usepackage{framed}
\usepackage[dvipsnames]{xcolor}
\usepackage{tcolorbox}
\usepackage{tikz}
\usepackage{booktabs}
\usepackage{tabularray}

\colorlet{LightGray}{White!90!Periwinkle}
\colorlet{LightOrange}{Orange!15}
\colorlet{LightGreen}{Green!15}

\newcommand{\HRule}[1]{\rule{\linewidth}{#1}}

\declaretheoremstyle[name=Theorem,]{thmsty}
\declaretheorem[style=thmsty,numberwithin=section]{theorem}
\tcolorboxenvironment{theorem}{colback=LightGray}

\declaretheoremstyle[name=Proposition,]{prosty}
\declaretheorem[style=prosty,numberlike=theorem]{proposition}
\tcolorboxenvironment{proposition}{colback=LightOrange}

\declaretheoremstyle[name=Principle,]{prcpsty}
\declaretheorem[style=prcpsty,numberlike=theorem]{principle}
\tcolorboxenvironment{principle}{colback=LightGreen}

\setstretch{1.2}
\geometry{
	textheight=9in,
	textwidth=5.5in,
	top=1in,
	headheight=12pt,
	headsep=25pt,
	footskip=30pt
}


\tikzstyle{terminator} = [rectangle,draw,text centered, rounded corners, minimum height=2em, minimum width=2em, draw=black, fill=red]

\tikzstyle{data}=[rectangle, draw, text centered, minimum height=2em, draw=black, fill=blue]

\tikzstyle{process} = [rectangle, draw, minimum height=1em, 
minimum width=3em, text centered, draw=black, fill=orange]

\tikzstyle{decision} = [diamond, draw, text centered, minimum height=1em, 
minimum width=3em, draw=black, fill=green]

% ------------------------------------------------------------------------------

\begin{document}
	
	% ------------------------------------------------------------------------------
	% Cover Page and ToC
	% ------------------------------------------------------------------------------
	
	\title{ \normalsize \textsc{}
		\\ [2.0cm]
		\HRule{1.5pt} \\
		\LARGE \textbf{\uppercase{EEG Peak Prediction}
			\HRule{2.0pt} \\ [0.6cm] \LARGE{Project} \vspace*{10\baselineskip}}
	}
	\date{}
	\author{\textbf{Author} \\ 
		Xinyu Qian and Satyarth Arora \\
		March 5, 2025}
	
	\maketitle
	\newpage
	
	\tableofcontents
	\newpage
	
	% ------------------------------------------------------------------------------
	\section{Abstract}
	This paper proposes a peak detection algorithm/model to recognize and label specific peaks found in an EEG signal. The model uses audiovisual interaction EEG data collected in Dr. Ozdamar's lab supplemented with EEG data found in online public databases.
	
	\section{Plan}
	\subsection{Peaks in an EEG signal}
	Our first plan of action is to extrapolate and explain in the most simplest terms, the different peaks observed in an EEG signal. We also will expand on meta properties about the peaks which will feed in to our algorithm as features.
	
	\subsection{Experiment Biases}
	As compositions of EEG signals are dependent on
	\begin{itemize}
		\item Equipment Setup: Type and number of electrodes used to record the data.
		\item Experiment Setup: Which part of the human brain is targeted for the experiment.
	\end{itemize}
	
	We will be describing the biases and providing justification on how our algorithm is bias independent/dependent.
	
	\subsection{Labeling and Validation of Peaks}
	
	To verify whether the peaks we are trying to find are accurate, we will be validating our EEG peak database from a domain expert (Dr. Ozdamar).
	
	\subsection{Strategies for Isolating Peaks}
	Next step is researching the state of the art on Isolation and peak extraction using signal processing or ML approaches.
	
	\subsection{Model/Algorithm Proposal}
	
	Upon doing the above, we will propose an architecture describing the best process (in our opinion) to achieve peak extraction and labelling.
	
	

		
	
	
	\newpage
	
	% ------------------------------------------------------------------------------
	% Reference and Cited Works
	% ------------------------------------------------------------------------------
	
	\bibliographystyle{IEEEtran}
	\bibliography{References.bib}
	
	% ------------------------------------------------------------------------------
	
\end{document}